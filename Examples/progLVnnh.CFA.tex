\documentclass{article}
\usepackage{tikz}
\usetikzlibrary{trees,shapes.geometric, positioning, arrows}

\begin{document}

\tikzstyle{state} = [circle, inner sep=2pt, radius =20pt, text centered, draw=black,  fill=blue!50]
\tikzstyle{point} = [circle, inner sep=0pt, minimum size =1pt,fill]
\tikzstyle{myarrow} = [->, >=latex]
\begin{tikzpicture}[node distance = 2cm, >=latex]

\matrix[row sep =5em,column sep=5em]{
\node(vert1)[state]{s1}; &\\
\node(vert9)[state]{s9}; &\\
\node(vert8)[state]{s8}; &\\
\node(vert7)[state]{s7}; &\\
\node(vert6)[state]{s6}; &\\
\node(vert5)[state]{s5}; &\node(vert4)[state]{s4};\\
\node(vert3)[state]{s3}; &\\
\node(vert2)[state]{s2}; &\\
};

\draw [myarrow]  (vert9)  --  node[sloped, near start,above]{x:=2} (vert8);
\draw [myarrow]  (vert8)  --  node[sloped, near start,above]{y:=4} (vert7);
\draw [myarrow]  (vert6)  -- node[sloped,above]{$y>x$} (vert5);
\draw [myarrow]  (vert6)  -- node[sloped,above]{$ !(y>x)$} (vert4);
\draw [myarrow]  (vert7)  --  node[sloped, near start,above]{x:=1} (vert6);
\draw [myarrow]  (vert5)  --  node[sloped, near start,above]{z:=y} (vert3);
\draw [myarrow]  (vert4)  |-  node[sloped, near start,above]{z:=y*y} (vert3);
\draw [myarrow]  (vert1)  --  node[sloped, near start,above]{} (vert9);
\draw [myarrow]  (vert3)  --  node[sloped, near start,above]{x:=z} (vert2);


\end{tikzpicture}
\end{document}

