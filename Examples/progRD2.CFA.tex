\documentclass{article}
\usepackage{tikz}
\usetikzlibrary{trees,shapes.geometric, positioning, arrows}

\begin{document}

\tikzstyle{state} = [circle, inner sep=2pt, radius =20pt, text centered, draw=black,  fill=blue!50]
\tikzstyle{point} = [circle, inner sep=0pt, minimum size =1pt,fill]
\tikzstyle{myarrow} = [->, >=latex]
\begin{tikzpicture}[node distance = 2cm, >=latex]

\matrix[row sep =5em,column sep=5em]{
 &\node(vert1)[state]{s1}; &\\
 &\node(vert9)[state]{s9}; &\\
 &\node(vert8)[state]{s8}; &\\
 &\node(vert5)[state]{s5}; &\node(vert7)[point]{};\\
 &\node(vert4)[state]{s4}; &\\
 &\node(vert3)[state]{s3}; &\\
\node(vert6)[point]{}; & &\\
 &\node(vert2)[state]{s2}; &\\
};

\draw [myarrow]  (vert9)  --  node[sloped, near start,above]{x:=5} (vert8);
\draw [myarrow]  (vert4)  --  node[sloped, near start,above]{y:=x*y} (vert3);
\draw [myarrow]  (vert5)  -- node[sloped,above]{$x>1$}  (vert4);
\draw [myarrow]  (vert5)  -- node[sloped,above]{$!(x>1)$}  (vert7);
\draw [myarrow]  (vert3) |-  node[sloped, near start,above]{x:=x-1} (vert6);
\draw [myarrow]  (vert6)  |-  (vert5);
\draw [myarrow]  (vert8)  --  node[sloped, near start,above]{y:=1} (vert5);
\draw [myarrow]  (vert1)  --  node[sloped, near start,above]{} (vert9);
\draw [myarrow]  (vert7)  |-  (vert2);


\end{tikzpicture}
\end{document}

